\chapter{Approach}
\label{chapter:approach}

Classification is the problem of determining to which of a group of categories an observation belongs to. 
Formally you can represent the classification problem as finding the correct class $y$  using the observation
 features $\bar{x}$ in the pair $(\bar{x},y)$, where $\bar{x}$ is the feature vector and $y$ the class.

\begin{equation}
	\bar{x} = < x_1,x_2,...,x_n >
\label{eq:featurevector}
\end{equation}
 
 Applied to the diagnosis/prognosis problem this is what is currently being done, where a model is created to discover
 $y_n$  from $\bar{x}_n$. As already mentioned the difference between a prognostic model and a diagnostic model is just the
 time between the present and the supposed time of $y_n$. 

That is why we are introducing time, the addition of time to the equation turns the pair $(\bar{x},y)$ into the triplet
 $(\bar{x}_i,y_i,t_i )$ where $t_i$ is a timestamp, $\bar{x}_i$ the feature vector in at time $t_i$ and $y_i$ the class also at time $t_i$.

 \begin{equation}
	\bar{x}_i = < x_{i1},x_{i2},...,x_{im} >
\label{eq:temporalvector}
\end{equation}
 
 In this case we have a sequence S of triplets, that is ordered and that can be used to predict a whole new 
 triplet $(\bar{x}_{n+1},y_{n+1},t_{n+1})$.
 
  \begin{equation}
	S= (\bar{x}_1,y_1,t_1 )… (\bar{x}_n,y_n,t_n ) \Rightarrow (\bar{x}_{n+1},y_{n+1},t_{n+1} )
\label{eq:sequence}
\end{equation}

Supposing $t_{n+1}$ is known our objective is to find $y_{n+1}$, the class in a future point in time.

In order to find $y_{n+1}$ we propose the following approaches:
\begin{enumerate}
\item{ To only find $y_{n+1}$

In this approach $y_{n+1}$ is found just based on the values for the feature vector $\bar{x}$ at each point of time, ignoring the values of $x_{n+1}$
  and do not trying to estimate them.
\begin{equation}
	y_{n+1} = f(x_{i} :1 \leq i \leq n)
\label{eq:approach1}
\end{equation}
  }
\item{ To find $(\bar{x}_{n+1},y_{n+1})$
	
Here  $y_{n+1}$ is found using diagnostic model on the feature vector $\bar{x}_{n+1}$ that, itself is found using various approaches:
	\begin{enumerate}
		\item{ $x_{n+1,k}=f(\bar{x}_j:1\leq j \leq n)$
		\begin{enumerate}
			\item{We can determine $x_{n+1,k}$ using the past values from the same feature, $x_{1,k},..., x_{n,k}$ . }
			\item{We can determine $x_{n+1,k}$ using the past values from all the features $\bar{x}_{1,k},..., \bar{x}_{n,k}$ and $y_1,...,y_n$.}
		\end{enumerate}
		}
	\end{enumerate}
}
\end{enumerate}

In the approach $i.$, only the values of one feature are used to predict the value of that feature, i.e. in order to predict $x_{n+1,i}$ only 
the values $x_{1,i},..., x_{n,i}$ will be used. 

This approach will be performed by using and/or adapting the techniques used in time series prediction. Here we need to find the technique 
that best deals with data that might not be numeric. This approach does not take into account the complex nature between features. 

On the approach $ii.$, all the features and eventually also the class are used when predicting each feature value. This approach will be 
performed using core data mining techniques, like decision trees and SVM, which capture the dependence relations between features.

These two approaches, described above, will work as baseline for comparison with the rest of the work. Where one represents the use of 
time but lacks in capturing the intrinsic relation between features and another the opposite, while capturing the dependence relation 
between features misses in the explicit use of time.

Another approach is to develop a representation for $x_{1_k},...,x_{n_k}$ that captures its’ evolution. Using that new representation of the 
data we can develop a new model that uses the relation between features as well as temporal patterns to correctly classify $y_{n+1}$.