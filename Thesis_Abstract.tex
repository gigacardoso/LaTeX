%%%%%%%%%%%%%%%%%%%%%%%%%%%%%%%%%%%%%%%%%%%%%%%%%%%%%%%%%%%%%%%%%%%%%%%%
%                                                                      %
%     File: Thesis_Abstract.tex                                        %
%     Tex Master: Thesis.tex                                           %
%                                                                      %
%     Author: Andre C. Marta                                           %
%     Last modified : 21 Jan 2011                                      %
%                                                                      %
%%%%%%%%%%%%%%%%%%%%%%%%%%%%%%%%%%%%%%%%%%%%%%%%%%%%%%%%%%%%%%%%%%%%%%%%

\section*{Abstract}

% Add entry in the table of contents as section
\addcontentsline{toc}{section}{Abstract}

The use of data mining techniques in healthcare has been noticing an increased relevance over the last few years, being applied with a variety of objectives, with the most common one being the automatic diagnostic process. In this process, data mining techniques have achieved interesting and successful results. However, when it comes to prognosis the same quality of results is not being achieved. We argue that this happens thanks to the inability of the used techniques to capture the inherent temporal dependencies present on the data. Specifically, the temporal evolution of a patient is not being taken into account when performing prognosis. In this paper, we propose a different approach, independent of the domain, to address this issue. We present our preliminary results on two different datasets that show an improvement in the overall precision of the prognosis.

\vfill

\textbf{\Large Keywords:} Prognosis, Classification, Temporal Dependencies

\cleardoublepage

