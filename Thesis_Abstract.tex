%%%%%%%%%%%%%%%%%%%%%%%%%%%%%%%%%%%%%%%%%%%%%%%%%%%%%%%%%%%%%%%%%%%%%%%%
%                                                                      %
%     File: Thesis_Abstract.tex                                        %
%     Tex Master: Thesis.tex                                           %
%                                                                      %
%     Author: Andre C. Marta                                           %
%     Last modified : 21 Jan 2011                                      %
%                                                                      %
%%%%%%%%%%%%%%%%%%%%%%%%%%%%%%%%%%%%%%%%%%%%%%%%%%%%%%%%%%%%%%%%%%%%%%%%

\section*{Abstract}

% Add entry in the table of contents as section
\addcontentsline{toc}{section}{Abstract}

The use of data mining techniques in the field of healthcare is still very much be-hind of where it needs to be, when a full
advantage is taken of the available data with help from data mining and analytical tools.

An area where this can be used is in the process of diagnosis and prognosis, where it can help the physicians decide the
 correct path to take with a certain patient, based on his probable prognosis. 

This is a process where a doctor takes into account an enormous amount of in-formation about the patients’ state as well
 as his evolution, fact that is not being used in the current approaches. This is what we propose to develop, a technique
 that can, generally, help in the process of prognosis by using the evolution of the patient over time when making the predictions.

We will start this document by mentioning the state of affairs in the field of prognosis, by quickly explaining the most
 commonly used techniques followed by use cases in various diseases. We end by describing our approach to the solution and 
 describe how we will validate our work. 


\vfill

\textbf{\Large Keywords:} keyword1, keyword2,...

\cleardoublepage

