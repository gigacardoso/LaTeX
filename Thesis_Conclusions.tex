%%%%%%%%%%%%%%%%%%%%%%%%%%%%%%%%%%%%%%%%%%%%%%%%%%%%%%%%%%%%%%%%%%%%%%%%
%                                                                      %
%     File: Thesis_Conclusions.tex                                     %
%     Tex Master: Thesis.tex                                           %
%                                                                      %
%     Author: Andre C. Marta                                           %
%     Last modified : 21 Jan 2011                                      %
%                                                                      %
%%%%%%%%%%%%%%%%%%%%%%%%%%%%%%%%%%%%%%%%%%%%%%%%%%%%%%%%%%%%%%%%%%%%%%%%

\chapter{Conclusions}
\label{chapter:conclusions}

There is a mismatch in the amount of data available in the field of healthcare and the data that is being used in order to gain knowledge.
 As it was shown in this dissertation, diagnosis and prognosis is a very relevant subject in the area of healthcare and that it has been subject 
 to some work in the past years. This work shows that no novel techniques are being introduced, being the same techniques used consistently throughout the years, and a visible lack of work improving on previous research with predicting models being developed independently. 

We also showed that the problem of prognosis is being tackled in the same way of diagnosis, not using the patients’ evolution over time
 in order to improve the results. 

In order to address this issue, we describe a novel approach that transforms prognosis into a diagnosis problem. This solution has two possible variants for the use of time on improving the results of a prognostic model. An univariate and a multivariate one where dependency relationships can be used to improve the final result.
The method was then evaluated and discussed using two different datasets in order to show its’ generalizability.

In this conclusion, we first highlight the main contributions of this work to the temporal pattern mining field and then discuss some directions for future work.

% ----------------------------------------------------------------------
\section{Achievements}
\label{section:achievements}

From the survey presented the lack of use of temporality in the prognosis problem can clearly be identified. 
Our contribution was the proposal and definition of an extensible method that uses the temporality of the data to improve the prognosis result. It is extensible because a lot different techniques or methods can be used in each step of the approach accordingly to the characteristics of the problem and the data.

This extensibility also introduces a level of generality and domain independence to our method that, based on the problem being tackled, allows us choose the various different techniques and configurations that may suit best the specific case. 


\hl[red]{Generalidade e independencia do dominio}


% ----------------------------------------------------------------------
\section{Future Work}
\label{section:future}

From the experimental comparison of the different approaches, over two distinct datasets (with different data characteristics,
 either from the medical and the data points of view), it is clear an improvement trend when using the temporal informed
 methods proposed. The shallow differences between the results of the estimation models, need to be deeply studied and other
 techniques (like Dynamic Bayesian networks) should be explored to enrich the estimation process. In either cases,
 the temporality of this kind of data should be considered as a core aspect of the prognosis.
 
Another possible variation to tackle the prognosis problem presented in this thesis would be to, instead of using the 
values that result from the estimation phase, like in the current approach, the model that represents the evolutionary trend of that
 feature would be used. Then the final classification would be performed on these models.

\cleardoublepage

