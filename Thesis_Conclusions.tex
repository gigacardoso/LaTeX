%%%%%%%%%%%%%%%%%%%%%%%%%%%%%%%%%%%%%%%%%%%%%%%%%%%%%%%%%%%%%%%%%%%%%%%%
%                                                                      %
%     File: Thesis_Conclusions.tex                                     %
%     Tex Master: Thesis.tex                                           %
%                                                                      %
%     Author: Andre C. Marta                                           %
%     Last modified : 21 Jan 2011                                      %
%                                                                      %
%%%%%%%%%%%%%%%%%%%%%%%%%%%%%%%%%%%%%%%%%%%%%%%%%%%%%%%%%%%%%%%%%%%%%%%%

\chapter{Conclusions}
\label{chapter:conclusions}

There is a mismatch in the amount of data available in the field of healthcare and the data that is being used in order to gain knowledge.
 As it was shown in this paper diagnosis and prognosis is a very relevant subject in the area of healthcare and that it has been subject 
 to some work in the past years. This work shows no evolution, being the techniques used consistently throughout the years and a visible
 lack of work improving on previous research with predicting models being developed independently. 

We also showed that the problem of prognosis is being tackled in the same way of diagnosis, not using the patients’ evolution over time
 in order to improve the results. 

In order to address this issue, we describe three possible solutions to the use of time in improving the results of a prognostic model.

We concluded by showing that this work will be validated in two different datasets in order to show its’ generalizability, and that
in each dataset the use of the usual classification metrics will be applied, along with a temporal pattern relevance analysis to show
 the relevance of time in the prognosis problem.



% ----------------------------------------------------------------------
\section{Achievements}
\label{section:achievements}

The major achievements of the present work...


% ----------------------------------------------------------------------
\section{Future Work}
\label{section:future}

From the experimental comparison of the different approaches, over two distinct datasets (with different data characteristics,
 either from the medical and the data points of view), it is clear an improvement trend when using the temporal informed
 methods proposed. The shallow differences between the results of the estimation models, need to be deeply studied and other
 techniques (like Dynamic Bayesian networks) should be explored to enrich the estimation process. In either cases,
 the temporality of this kind of data should be considered as a core aspect of the prognosis.
 
\hl{Another possible variation to tackle the prognosis problem presented in this thesis would be that, instead of using the 
values that result from the estimation phase, in the current approach, the model that represents the evolutionary trend of that
 feature would be used. Then the final classification would be performed on these models.}

\cleardoublepage

