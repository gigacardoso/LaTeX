\chapter{Approach}
\label{chapter:approach}

Classification is the problem of determining to which of a group of categories an observation belongs to. 
Formally you can represent the classification problem as finding the correct class $y$  using the observation
 features $\bar{x}$ in the pair $(\bar{x},y)$, where $\bar{x}$ is the feature vector and $y$ the class.

\begin{equation}
	\bar{x} = < x_1,x_2,...,x_n >
\label{eq:featurevector}
\end{equation}
 
 Applied to the diagnosis/prognosis problem this is what is currently being done, where a model is created to discover
 $y_n$  from $\bar{x}_n$. As already mentioned the difference between a prognostic model and a diagnostic model is just the
 time between the present and the supposed time of $y_n$. 

That is why we are introducing time, the addition of time to the equation turns the pair $(\bar{x},y)$ into the triplet
 $(\bar{x}_i,y_i,t_i )$ where $t_i$ is a timestamp, $\bar{x}_i$ the feature vector in at time $t_i$ and $y_i$ the class also at time $t_i$.

 \begin{equation}
	\bar{x}_i = < x_{i1},x_{i2},...,x_{im} >
\label{eq:temporalvector}
\end{equation}
 
 In this case we have a sequence S of triplets, that is ordered and that can be used to predict a whole new 
 triplet $(\bar{x}_{n+1},y_{n+1},t_{n+1})$.
 
  \begin{equation}
	S= (\bar{x}_1,y_1,t_1 )… (\bar{x}_n,y_n,t_n ) \Rightarrow (\bar{x}_{n+1},y_{n+1},t_{n+1} )
\label{eq:sequence}
\end{equation}

Supposing $t_{n+1}$ is known our objective is to find $y_{n+1}$, the class in a future point in time.

In order to find $y_{n+1}$ we propose the following approaches:
\begin{enumerate}
\item{ To only find $y_{n+1}$

In this approach $y_{n+1}$ is found just based on the values for the feature vector $\bar{x}$ at each point of time, ignoring the values of $x_{n+1}$
  and do not trying to estimate them.
\begin{equation}
	y_{n+1} = f(x_{i} :1 \leq i \leq n)
\label{eq:approach1}
\end{equation}
  }
\item{ To find $(\bar{x}_{n+1},y_{n+1})$
	
Here  $y_{n+1}$ is found using diagnostic model on the feature vector $\bar{x}_{n+1}$ that, itself is found using various approaches:
	\begin{enumerate}
		\item{ $x_{n+1,k}=f(\bar{x}_j:1\leq j \leq n)$
		\begin{enumerate}
			\item{We can determine $x_{n+1,k}$ using the past values from the same feature, $x_{1,k},..., x_{n,k}$ . }
			\item{We can determine $x_{n+1,k}$ using the past values from all the features $\bar{x}_{1,k},..., \bar{x}_{n,k}$ and $y_1,...,y_n$.}
		\end{enumerate}
		}
	\end{enumerate}
}
\end{enumerate}

In the approach $i.$, only the values of one feature are used to predict the value of that feature, i.e. in order to predict $x_{n+1,i}$ only 
the values $x_{1,i},..., x_{n,i}$ will be used. 

This approach will be performed by using and/or adapting the techniques used in time series prediction. Here we need to find the technique 
that best deals with data that might not be numeric. This approach does not take into account the complex nature between features. 

On the approach $ii.$, all the features and eventually also the class are used when predicting each feature value. This approach will be 
performed using core data mining techniques, like decision trees and SVM, which capture the dependence relations between features.

These two approaches, described above, will work as baseline for comparison with the rest of the work. Where one represents the use of 
time but lacks in capturing the intrinsic relation between features and another the opposite, while capturing the dependence relation 
between features misses in the explicit use of time.

Another approach is to develop a representation for $x_{1_k},...,x_{n_k}$ that captures its’ evolution. Using that new representation of the 
data we can develop a new model that uses the relation between features as well as temporal patterns to correctly classify $y_{n+1}$.

\section{example}
\label{section:example}

Let’s assume we have clinical and laboratorial data for a set of patients with Alzheimer’s disease. The data itself could be 
represented like in \ref{tab:diagnostic}. 

\begin{table}[h]
\begin{center}
\begin{tabular}{ccccccc}
\textbf{Patient ID} & \textbf{Gender} & \textbf{Age} & \textbf{HBP} & \textbf{LBP} & \textbf{Degree of Progression} & \textbf{Time Step} \\ \hline
25         & M      & 65  & 160 & 88  & 1                     & 0         \\
25         & M      & 65  & 140 & 90  & 1                     & 1         \\
25         & M      & 65  & 138 & 85  & 1                     & 2         \\
25         & M      & 65  & 134 & 81  & 1                     & 3         \\
25         & M      & 65  & 141 & 88  & 2                     & 4         \\ \hline
\end{tabular}
\caption{Table 1}
\label{tab:diagnostic}
\end{center}
\end{table}
 
As seen in \ref{tab:diagnostic} we have $N$ time steps of data, in this case 5, we are going to use those in order to perform prognosis. As described
 in the last section we have 2 approaches to doing that, in the first we only use the past values of a feature to predict the future
 value of that feature, that is, for example with the high blood pressure, we can use HBP in times $0,1,2,3,4$ in order to predict $5$, and
 the same for every other feature.

In \ref{tab:approach1} we can see the data what would be used in order to predict HBP at time $5$.

\begin{table}[h]
\begin{center}
\begin{tabular}{cccccc}
\textbf{HBP\_0} & \textbf{HBP\_1} & \textbf{HBP\_2} & \textbf{HBP\_3} & \textbf{HBP\_4} & \textbf{HBP\_5} \\ \hline
160             & 140             & 138             & 134             & 141             & ?					\\ \hline        
\end{tabular}
\caption{Table 2}
\label{tab:approach1}
\end{center}
\end{table} 

The second option we would use every feature value to predict each one. In this case we would use the time steps $0$ to $4$ of every feature, as can be seen in \ref{tab:approach2} to predict
 HBP $5$, as well as every other feature.
 
 \begin{table}[h]
 \begin{center}
\begin{tabular}{ccccccccccccc}
\textbf{Patient ID} & \textbf{Gender} & \textbf{Age\_0} & \textbf{HBP\_0} & \textbf{LBP\_0} & \textbf{Age\_1} & \textbf{HBP\_1} & \textbf{LBP\_1} & \textbf{…} & \textbf{Age\_4} & \textbf{HBP\_4} & \textbf{LBP\_4} & \textbf{HBP\_5} \\ \hline
25                  & M               & 65              & 160             & 88              & 65              & 140             & 90              &            & 65              & 141             & 88              & ?              \\ \hline
\end{tabular}
\caption{Table 3}
\label{tab:approach2}
\end{center}
\end{table}

At the end of this two approached the result is the same, which is a complete data row at time step 5.

Using that predicted data on a diagnostic model that was previously trained on data like in \ref{tab:diagnostic}.
 We would get the final Prognosis.
\cleardoublepage
 
\chapter{Approach2}
\label{chapter:approach2}


\hl{In the medical context, diagnosis is the use of patient’s data, demographic and clinical, in order to understand 
and classify the current health condition of a patient \cite{Steyerberg2005}. From a formal point of view, and in the computer-based 
context, let $A$ be a set of variables (either known as attributes) and $C$ a set of possible classes. Given an instance
 $\bar{x}_i$ described by a set of $m$ variables from $A$, say $\bar{x}_i = (x_{i,1},... ,x_{i,m})$, the goal is to discover the most
 probable value $y_i$, which corresponds to its class or status, with $y_i \in C$, as in \ref{eq:classification}.}

\begin{equation}
	\bar{x}_i = (x_{i,1},... ,x_{i,m}) \rightarrow y_i
\label{eq:classification}
\end{equation}


\hl{In a classification context, this is done in two steps: first by producing a classification model $M_D$, based on a set of known
 pairs $(x_i, y_i)$ – the training dataset, and second, by applying the discovered model to each instance to classify.}

\hl{On the other hand, \emph{prognosis} is the foreseeing or prediction of the risk or probability of a certain health event happening,
 in the future, using the clinical and non-clinical data. It is the medical prediction of how the pair patient disease is going
 to evolve in a specified period of time.}

\hl{To do this prognosis, a physician will use data that relates the patient to a certain part of the population, i.e. demographic data,
 as well as the patient’s and patient’s family clinical history. This means that the evolution of the patient is important in the
 prediction of his next state. Simply putting, if a patient is showing improvement in a certain factor that is responsible for
 some disease, it is more probable that his prognosis related to that disease is better than if it the patient had the same 
 value but that factor was deteriorating.}

\hl{As previously stated, in the process of making a prognosis a physician uses the medical history of a patient. This includes
 the different states a patient has been in the form of various clinical analyses he had done in different points over time.
 The need to use this sequential information shows the utmost importance that time has when predicting someone’s survivability,
 risk of recurrence.}

\hl{Considering all of this, then the prognosis task can be formalized as follows:}

\hl{Let a patient be represented by a sequence of pairs, $(\bar{x}_i^1,y_i^1 ),...,(\bar{x}_i^n,y_i^n )$ , then the goal is to predict
 his $y_i^{n+1}$ value – equation \ref{eq:goal1}. Note that the different values for $y_i^t$ may be observable (available) or non-observable 
 at time instant $t$ for instance $i$.}


\begin{equation}
	(\bar{x}_i^1,y_i^1 ),...,(\bar{x}_i^n,y_i^n ) \rightarrow y_i^{n+1}
\label{eq:goal1}
\end{equation}


The traditional classification approach has been applied to prognosis with modest success, as seen above. In all described 
cases, the evolution of single variables was not explored, and actually, the different time instances of their values were
 addressed separately, ignoring any possible hidden structure, in the majority of approaches. On the other hand, the analysis 
 of time series is applied to predict the next outcome of a single variable.

By recognizing that estimation may be used to fill unseen variable outcomes, which in turn may be used to improve classifiers
 accuracy, as in asap classifiers \cite{Antunes2010}, we propose to transform the prognosis into a diagnosis task, by estimating the values
 of the variables that constitute the snapshot in the future point in time.

Formally, let $A$ be a set of attributes, $C$ be a set of possible classes and $n$ be the number of observations. Let the $t^{th}$ observation,
 described by $m$ variables from $A$, be the pair given by $(\bar{x}_i^t,y_i^t)=(x_{i1}^t,...,x_{im}^t,y_i^t)$ that says that at observation $t$
 the instance is described by $x_i^t$ (the observable values) and classified as $y_i^t \in C$(the predicted value). Given an instance 
 described by an ordered set of $n$ observations, the goal is to predict the ${n+1}^{th}$ observation, as in equation \ref{eq:goal}.

\begin{equation}
	(\bar{x}_i^1,y_i^1 ),...,(\bar{x}_i^n,y_i^n ) \rightarrow y_i^{n+1}
\label{eq:goal}
\end{equation}

The difference to the definition \ref{eq:goal1} is the need to predict the entire ${n+1}^{th}$ observation, not only the predicted 
value $y_i^{n+1}$. Indeed, if there is a model $M_D$, that from observable values is able to determine the predicted value, it is 
enough to estimate the observable values in the ${n+1}^{th}$ observation, and from them to predict the predicted value. This model $M_D$ is
 just a simple diagnosis model as in equation \ref{eq:classification}.

 According to this formulation, a prognosis model, $M_P$, is then the composition of several models: one estimation model $M_{Ek}$ per each 
 observable variable $X_k$ and a diagnosis model $M_D$ able to predict the class given an observation, as in equation \ref{eq:prognosis}, where $n$ corresponds 
 to the number of available observations and $m$ the number of variables for describing each observation.

\begin{equation}
	M_P ((\bar{x}_i^1,y_i^1 )\space...\space(\bar{x}_i^n,y_i^n )) = M_D (M_{E1} (\bar{x}_i^1 \space...\space \bar{x}_i^n ) \space...\space M_{Em}
	(\bar{x}_i^1 \space...\space \bar{x}_i^n ))
\label{eq:prognosis}
\end{equation}

By transforming the prognosis problem into a diagnosis task, the challenge becomes to be able to estimate the observation in the
 time point to predict, which translates into the definition of the estimation models per each observable variable.

As stated above, the art of prognosis is based on the analysis of the evolution of the different variables along time. Therefore,
 estimation models should be able to recognize verified evolution trends in the estimation of future values. 

In this manner, we propose that an estimation model for a single variable $X_k$, say $M_{Ek}$ should be a function from a sequence of the
 observed values to an $X_k$ value. In particular, we propose two different approaches: the \emph{univariate-based} and 
 the \emph{multivariate-based estimations}.

A \emph{\underline{u}ni\underline{v}ariate-based model} for variable $X_k\space(UvE)$ is a function from a sequence of $n$
 values of $X_k$ to its next value, $x_k^{n+1}$, as
 in equation \ref{eq:uve}, where $Dom_{Xk}$ represents the domain of variable $X_k$. These models only explore the individual values of a 
 variable, ignoring any influence from other variables.
 
\begin{eqnarray}
	M_{UvEk}:[Dom_{Xk}]^n \rightarrow Dom_{Xk}        \\ \label{eq:uve}
	M_{UvEk}(x_k^1 \space...\space x_k^n) = x_k^{n+1} \nonumber
\end{eqnarray}

On the counterpart, a \emph{\underline{m}ulti\underline{v}ariate-based model} for variable $X_k \space (MvE)$ is a function from a sequence 
of $n$ vectors of $m$ variables, including $X_k$, to its next value, $x_k^{n+1}$ – see equation \ref{eq:mve}.

\begin{eqnarray}
	M_{MvEk}:[Dom_{X1} \times ... \times Dom_{Xm}]^n \rightarrow Dom_{Xk}  \\ \label{eq:mve}
	M_{MvEk}(\bar{x}^1 \space...\space \bar{x}^n) = x_k^{n+1} \nonumber
\end{eqnarray}

By receiving a sequence of multi-values, recorded along $n$ observations, multivariate estimator is able to contemplate the interdependencies
 among the different values, and having more informed inputs, is expected to output better estimations.
 
 \section{Algorithm}
 \label{section:algorithm}
 
% Insert the algorithm
\begin{algorithm}
\caption{Pseudocode for Univariate Estimation training}
\label{alg:uve}
\begin{algorithmic}[1]
\Procedure{UnivariateEstimation}{Dataset $D$, Function $alg_{class}$, int $\rho$}
	\State $// D$ – the training dataset with
	\State $// \qquad\qquad\qquad D = \{(\bar{x}_i^t,y_i^t ): \forall i,t:1 \leq i \leq |D| \wedge 1 \leq t \leq n\} $
	\State $// alg_{class}$ – the training algorithm
	\State $// \rho$ – the number of observations to use
    \State $A \leftarrow \{$the set of attributes describing $D\}$
	\Statex
	\State $//$ Training each estimation model
    \For {each variable $X_k$ in $A$}
        \State $D_k \leftarrow \pi_{Xk} (D) = \{(x_{ik}^{n-\rho},...,x_{ik}^n ): \forall \bar{x}_i \in D\} $
		\State $M_{Ek} \leftarrow alg_{class}(D_k) $
	\EndFor
	\Statex
	\State $//$ Estimating n+1 snapshot
	\For {each variable $X_i$ in $D$}
		\For {each variable $X_k$ in $A$}
			\State $x_{ik}^{n+1} \leftarrow M_{Ek} (x_{ik}^{n-\rho},...,x_{ik}^n )$
			\State $ D^{n+1} \leftarrow D^{n+1} \cup \{(x_{i1}^{n+1},...,x_{im}^{n+1},y_i^{n+1} )\}$
		\EndFor
	\EndFor
	\Statex
	\State $//$ Train the diagnosis model
	\State $M_D \leftarrow  alg_{class}(D^{n+1})$
	\Statex
	\State $//$ Output the composition of models
	\State Return $M_D \circ (M_{E1},...,M_{Ek})$
\EndProcedure
\Statex
\end{algorithmic}
\end{algorithm}

Note, that the dataset has to be composed of records containing $n$ snapshots, as described before, and $\rho$ has to be less or equal to $n$. 
In terms of the classification training algorithm, it should be any tabular one, like a decision tree learner, an algorithm for
 training neural networks or just naïve Bayes.

The difference between the models is on the creation of the estimation models (line 9), in particular on the creation of the 
training dataset for each variable. While for univariate model, it consists on the projection of $D$ in relation to each $X_k$, the 
multivariate model uses the entire set of variables. In both cases, $\rho$ corresponds to the number of snapshots to keep in the dataset.
 Since, it is usual that the instants more significant for determining the next value are the previous ones, only the last $\rho$ snapshots are used.

After training the estimation model for each variable, the diagnosis model is learnt from the estimated snapshot for instant
 $n+1$ and the known class label. Then, the algorithm outputs the model resulting from the composition of the different
 estimators and the diagnosis model learnt from the estimated values (line 22).


\cleardoublepage