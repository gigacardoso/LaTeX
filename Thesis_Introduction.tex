%%%%%%%%%%%%%%%%%%%%%%%%%%%%%%%%%%%%%%%%%%%%%%%%%%%%%%%%%%%%%%%%%%%%%%%%
%                                                                      %
%     File: Thesis_Introduction.tex                                    %
%     Tex Master: Thesis.tex                                           %
%                                                                      %
%     Author: Andre C. Marta                                           %
%     Last modified : 28 Feb 2014                                      %
%                                                                      %
%%%%%%%%%%%%%%%%%%%%%%%%%%%%%%%%%%%%%%%%%%%%%%%%%%%%%%%%%%%%%%%%%%%%%%%%
%\usepackage{soul}
\newcommand{\hl}[2][yellow]{\textcolor{#1}{#2}}

\chapter{Introduction} 
\label{chapter:introduction}

%An enormous amount of data exist in the field of healthcare while the amount of knowledge that is being gathered from that data is still very limited. The data can be used to teach us new causal relations between symptoms and diseases as well a variety of other relevant information. 

%Another way where the use of this data can be very helpful is in the difficult task of prognosis, which helps patients and their carers decide and plan the best course of action to take, taking into account the most probable outcome in the patients dis-ease. Nowadays in data mining, prognosis is  being done the exactly same way as diagnoses using the current state of the patient, but not taking into account his evolution.

%To fix this problem of not using the evolution of the state of the patient we pro-pose an approach that includes this sequence of states by  incorporating time in the construction of the prediction model.

% We will start by describing what is diagnosis and prognosis and how data mining techniques have been used to help in these areas of healthcare.  We show the state of affairs in terms of prognosis, how it is performed in a variety of diseases showing that the techniques all circle around  the same. We finish by proposing a different approach for the prognosis problem and discuss how these approaches will be validated. 

%----------------------------------------------------------------------

The role of data analysis in healthcare has gained more attention, as available mining techniques have achieved higher levels of maturity.
 In particular, classification methods become to play a decisive role when applied to clinical trials, by providing high quality external 
 evidence to support evidence-based medicine \cite{Sackett1996}. The rigorous metrics available to evaluate the confidence about the collected evidence 
 on those trials, allied to the variety of techniques suited to different kinds of data, revealed to be fundamental to keep expertise 
 up-to-date and available worldwide.

Despite the success of those techniques, they are mostly appropriate to analyze tabular data, described by a set of independent variables.
 Actually, we can see this kind of data as a static snapshot of the status of some entity, which is completely suited to represent patient
 records collected during their diagnosing process. On the other hand, prognosis may be seen as the prediction of an outcome in a future 
 instant, considering all available data collected along time. In this manner, we may think of prognosis as the task of predicting an
 outcome, given a set of time-ordered snapshots. While in a single snapshot, methods may assume some level of independency among
 variables, this assumption is clearly unlikely in a set of snapshots, where the same variable is measured along different instants of time. 

Actually, and despite this dependency among snapshots, a large number of classification-based approaches have been proposed for prognosis
 (see \cite{Endo2008}, \cite{Paradise2009}, \cite{Zhou2011}, for example). In our opinion, the results achieved through them have been
 impaired due to the dependency among the  different values for the same variable along time. 

In this dissertation, we argue that the simple prediction of the prognosis outcome by traditional classification methods, given a set of
 snapshots, can be significantly improved by exploring the temporal relations, or evolution verified in each variable that compose 
 the snapshots. In order to validate our claim, we formalize the problem addressed, and present an approach to take those dependencies
 into account in the process of outcome prediction. We also perform a comparative analysis between two techniques used to estimate 
 the future values of some features.

After the formalization of the prognosis problem, we review a set of case studies on several different diseases, with the most 
well-known classification techniques (chapter \ref{chapter:rw}). In chapter \ref{chapter:approach} we describe our approach, and propose two distinct implementations of it, followed by a description of some experiments that compare the accuracy of both traditional classifiers and our approach 
 using two different techniques for the estimation phase (chapter \ref{chapter:results}). The dissertation concludes with a discussion of the improvements achieved, the issues constraining those improvements and proposing some guidelines for the next steps (chapter \ref{chapter:conclusions}).

\cleardoublepage